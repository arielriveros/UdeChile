%%%% nSystemDoc.tex

\documentstyle[fullpage,spanish]{article}
\parindent 0in
\begin{document}      

\begin{center}
{\large\bf CC41B\,: Sistemas Operativos}\\
{\bf Prof.: Luis Mateu.}\\
{\bf Autor: Jaime Catal'an.}\\
\bigskip
{\large \bf El Nano System\,: nSystem}
\end{center}

\bigskip
{\tt nSystem} es un sistema de procesos livianos para Unix con fines
pedag'ogicos.  El sistema consiste de unos cientos de l'ineas de
c'odigo en C (la mitad son comentarios) que implementan\,:

\begin{itemize}

\item Creaci'on y destrucci'on de procesos livianos (tareas).  Estos procesos
comparten un mismo espacio de direcciones (est'an dentro de un solo
proceso UNIX).

\item Paso de Mensajes para la sincronizaci'on de tareas.

\item Entrada/Salida no bloqueante para el proceso UNIX.

\item Un scheduler muy simple. Permite implementar administraci'on preemptive
y non-preemtive.


\end{itemize}


\subsubsection*{Archivos fuentes}

En \verb+cipres:~cc41b/nSystem/+ Ud. encontrar'a los fuentes de nSystem.
Los archivos contienen\,:

\begin{itemize}

\item {\tt nSystem.h}\,: Encabezados que debe incluir toda aplicaci'on
que usa nSystem.  Estudiar cuidadosamente.

\item {\tt nSysimp.h}\,: Encabezados con las estructuras de datos que
usa nSystem internamente.

\item {\tt nProcess.c}\,: Implementaci'on de nSystem.

\item {\tt nMsg.c}\,: Implementaci'on mensajes entre tareas.

\item {\tt nSem.c}\,: Implementaci'on de sem'aforos.

\item {\tt nMonitor.c}\,: Implementaci'on de monitores.

\item {\tt nIO.c}\,: E/S bloqueante para la tarea, pero no para el resto.

\item {\tt nMain.c}\,: El main del programa C.

\item {\tt nOther.c}\,: Procedimientos varios. (ej. nPrintf, nMalloc, etc.) 

\item {\tt nQueue.c}\,: Manejo de colas.

\item {\tt nDep.c}\,: Rutinas varias que hacen poco, pero son
fuertemente dependientes de Unix.

\item {\tt nStack-sparc.s}\,: Escrito en assembler, contiene procedimientos
para cambiar la pila en la arquitectura sparc.

\item {\tt nStack-i386.s}\,: Escrito en assembler, contiene procedimientos
para cambiar la pila en la arquitectura intel 386.

\end{itemize}

\subsubsection*{ Estado actual de nSystem }

A continuaci'on se detallan los procedimientos p'ublicos de nSystem.

\begin{itemize}

\item {\tt int nMain(/* int argc, char *argv[] */)}\,:  Este procedimiento es
provisto por el programador y es invocado al comenzar la ejecuci'on.
El retorno de este procemiento termina prematuramente todas las tareas
pendientes. (No siempre sera necesario colocar argc y argv, por ello
los argumentos estan comentados).

\end{itemize}

Creaci'on y Muerte de tareas\,:

\begin{itemize}

\item {\tt nTask nEmitTask( int (*proc)(), parametro1 ... parametro14 )}\,:
Emite una tarea que ejecuta el procedimiento {\tt proc}. Acepta un
m'aximo de 14 par'ametros (enteros o punteros).

\item
{\tt void nExitTask(int rc)}\,:
Termina la ejecuci'on de una tarea,
{\tt rc} es el c'odigo de retorno de la tarea.

\item
{\tt int   nWaitTask(nTask task)}\,:
Espera a que una tarea termine,
entrega el c'odigo de retorno dado a {\tt nExitTask}.

\item
{\tt void nExitSystem(int rc)}\,:
Termina la ejecuci'on de todas las tareas (shutdown del proceso Unix),
{\tt rc} es el c'odigo de retorno del proceso UNIX.

\end{itemize}

Par'ametros para las tareas\,:

\begin{itemize}

\item
{\tt void nSetStackSize(int new\_size)}\,:
Define el tama'no del stack de las tareas que se emitan a continuaci'on.

\item
{\tt void nSetTimeSlice(int slice)}\,: Tama'no de la tajada de tiempo
para la administraci'on {\em Round--Robin} (preemptive) ({\tt
slice} esta en miliseg).  Degenera en {\em FCFS} (non--preemptive) si {\tt
slice} es cero, lo que es muy 'util para depurar programas. El valor
por omisi'on de la tajada de tiempo es cero.

\item
{\tt void nSetTaskName( /* char *format, <args> ... */ )}\,:
Asigna el nombre de la tarea que la invoca.
El formato y los par'ametros que recibe son an'alogos a los de {\tt printf}.

\end{itemize}

Paso de Mensajes :

\begin{itemize}

\item {\tt int nSend(nTask task, void *msg)}\,: Env'ia el mensaje {\tt
msg} a la tarea {\tt task}.  Un mensaje consiste en un puntero a un
'area de datos de cualquier tama'no.  El emisor se queda bloqueado
hasta que se le haga {\tt nReply}. {\tt nSend} retorna un entero
especificado en {\tt nReply}.

\item {\tt void *nReceive(nTask *ptask, int max\_delay)}\,:
Recibe un mensaje proveniente de cualquier tarea.  La identificaci'on
del emisor queda en {\tt *ptask}.  El mensaje retornado por {\tt
nReceive} es un puntero a un 'area que ha sido posiblemente creada en
la pila del emisor.  Dado que el emisor contin'ua bloqueado hasta que
se haga {\tt nReply}, el receptor puede accesar libremente esta 'area
sin peligro de que sea destruida por el emisor.

La tarea que la invoca queda bloqueada por {\tt max\_delay} miliseg,
esperando un mensaje. Si el per'iodo finaliza sin que llegue alguno,
se retorna {\tt NULL} y {\tt *ptask} queda {\tt NULL}.  Si {\tt
max\_delay} es 0, la tarea no se bloquea (nReceive retorna de
inmediato). Si {\tt max\_delay} es -1, la tarea espera indefinidamente
la llegada de un mensaje.


\item {\tt void nReply(nTask task, int rc)}\,: Responde un mensaje
enviado por {\tt task} usando {\tt nSend}.  Desde ese instante, el
receptor no puede accesar la informaci'on contenida en el mensaje que
hab'ia sido enviado, ya que el emisor podr'ia destruirlo.  El valor
{\tt rc} es el c'odigo de retorno para el emisor. nReply no se
bloquea.

\end{itemize}

Entrada y Salida\,:\\

Estas funciones son equivalentes a open, close, read y write en
UNIX. Sus par'ametros son an'alogos a los de las de UNIX.
Las ``nano'' funciones son no bloqueantes para el proceso UNIX,
s'olo bloquean la tarea que las invoca. 

\begin{itemize}

\item {\tt int nOpen(char *path, int flags, int mode)}\,: Abre un archivo.

\item {\tt int nClose(int fd) }\,: Cierra un archivo.

\item {\tt int nRead(int fd, char *buf, int nbyte) }\,: Lee de un archivo.

\item {\tt int nWrite(int fd, char *buf, int nbyte) }\,: Escribe en un archivo.

\end{itemize}

Servicios Miscel'aneos\,:

\begin{itemize}

\item {\tt nTask nCurrentTask(void) }\,:
Entrega el identificador de la tarea que la invoca.

\item {\tt int nGetTime(void) }\,:
Entrega la ``hora'' en miliseg.

\item
{\tt void *nMalloc(int size)}\,:
Es un {\tt malloc} ininterrumpible.

\item
{\tt void nFree(void *ptr)}\,:
Es un {\tt free} ininterrumpible.


\item
{\tt void nFatalError(/* char *procname, char *format, ...*/ )}\,:
Escribe salida formateada en la salida est'andar de errores y termina
la ejecuci'on (del proceso Unix).  El formato y los par'ametros que
recibe son an'alogos a los de {\tt printf}.

\item
{\tt void nPrintf(/* char *format, ... */)}\,:
Es un {\tt printf} solo bloqueante para la tarea que lo invoca.

\item
{\tt void nFprintf(/* int fd, char *format, ... */)}\,: Es ``como'' un
{\tt fprintf}, solo bloqueante para la tarea que lo invoca, pero
recibe un fd, no un FILE *.

\item
{\tt void nSetNonBlocking()}\,: Coloca la E/S est'andar en modo no
bloqueante.  De esta forma, al leer la entrada esta'ndar s'olo se
bloquear'a la tarea lectora y no el resto de las tareas.

\end{itemize}

\end{document}
